\chapter{Moderation \& Informelle Führung}

\section{Informelle Führung [MEP]}

\textbf{Führung} = Mit Leuten Ziele erreichen \\

\textbf{Formelle Führung} In einer Abteilung ist der Abteilungsleiter gegenüber den Mitarbeiter der disziplinarische Vorgesetzte. Er hat die organisatorische Befehlsgewalt.\\

\textbf{Informelle Führung} Wenn der Chef keine formelle Machtbefugnisse hat. Beispielsweise findet man dies in Projektteams mit Mitarbeitern aus verschiedenen Abteilungen. Informell führt man durch moderieren und teilnehmen - ohne Macht!\\

Der wichtigste Grundsatz für den informellen Führer im Umgang mit seinem Team ist folgender: Fragen statt Sagen! Wer fragt, führt - wer führt, gewinnt. Man muss seinem Team geistig voraus sein, sodass zum richtigen Zeitpunkt die richtigen Fragen gestellt werden.

\section{Teamwork}

\begin{itemize}
	\item jeder kommt gut vorbereitet
	\item jeder ordnet seine eigenen Interessen der Allgemeinheit unter
	\item es gibt keine Machtstrukturen
	\item jeder ist immer sachlich -- Gefühle haben im Geschäftsleben keinen Platz
	\item alle sind immer gut drauf, die Energie ist immer gleich hoch
	\item Konflikte lassen sich durch ein Gespräch unter 4 Augen bereinigen
\end{itemize}

\section{Ablauf einer Gruppen-Interaktion}

\begin{enumerate}
	\item Begrüssung durch Moderator
	\item Intro
		\begin{enumerate}
			\item Aufgabe, Ziel, erwartete Ergebnisse
			\item Ablauf
			\item Spielregeln
			\item Doku (Fotoprotokoll)
		\end{enumerate}
	\item Gruppenrunde (vorstellen)
	\item Informationsgleichstand herstellen
	\item Thema bearbeiten
	\item Handlungsorientierung (ToDo Liste)
	\item Abschluss (Offene Punkte, Feedback)
\end{enumerate}

\textbf{Bei einem Meeting sind folgende Punkt unerlässlich! [MEP]}
\begin{enumerate}
	\item Klares Ziel definieren, welche Ergebnisse müssen am Schluss vorliegen.
	\item Gruppenrunde ziemlich am Anfang des Meetings. Jeder sagt wenigsten einen Satz.
	\item Abschluss mit ToDo-Liste und kurze Feedback-Runde.
\end{enumerate}

\section{Drehbuch Workshop}

\begin{tabbing}
	\hspace{6cm}\=\kill
	\textbf{Start:} \> -- Mit Begrüssungskaffee/-tee beginnen (=Puffer) \\ \\
	\textbf{Gruppenrunde:} \> -- Immer! \\
						   \> -- Kann sehr kurz sein \\
						   \> -- Möglichkeiten: Steckbrief (Plakat/Folie), Gruppenspiegel usw. \\ \\
	\textbf{Mehrere Themen:} \> -- Liste machen \\ 
							\> -- gemeinsam Priorität bestimmen und Zeit vereinbaren \\
							\> -- Zeit überwachen, rückfragen ggf. Ziel anpassen \\ \\
	\textbf{Informations-Gleichstand:} \> -- Immer! \\ 
									   \> -- auch vor-verteilte Info kurz repetieren \\ \\
	\textbf{Themenspeicher:} \> -- Bei regelmässigen Meetings offene Themen festhalten \\ \\
	\textbf{Feedback:} \> -- Immer! \\
					   \> -- Zur Moderation, Zielerreichung, Arbeitsweise, usw. \\ \\
	\textbf{Pflicht-Plakate:} \> -- Spielregeln \\
							  \> -- Offene Punkte \\
							  \> -- Blitzlichtregeln
\end{tabbing}

\section{Meine Rolle als WS-Leiter}

Ich bereite den Arbeitsprozess der Gruppe methodisch vor und begleite ihn:
\begin{itemize}
	\item Drehbuch vorbereiten
	\item Umgebung gestalten
	\item Ergebnisse dokumentieren (Visualisieren, Fotoprotokoll)
\end{itemize}
Ich kümmere mich darum, dass die Gruppe (möglichst) immer am Punkt ihrer optimalen Leistungsfähigkeit arbeiten kann. Für das inhaltliche Ergebnis ist die Gruppe selbst verantwortlich!

\section{Workshop Spielregeln}

\begin{itemize}
	\item Jeder hier ist wichtig
	\item Kurz und Prägnant sprechen -- Zuhören und ausreden lassen
	\item Fragen statt interpretieren
	\item \emph{Ich} statt \emph{wir} oder \emph{man}
	\item Jeder ist für sich selbst verantwortlich
	\item Störungen haben Vorrang
\end{itemize}

\section{Vorbereitung eines Workshops}

\begin{tabbing}
	\hspace{3cm}\=\kill
	\textbf{Titel:}	\> -- Wenn nicht vom Auftraggeber definiert, selbst erfinden\\ 
					\> -- Keine Moderation ohne Titel!\\ \\
	\textbf{Zeitrahmen:} \> -- überprüfen, ob realistisch für die Zielerreichung mit einer Gruppe \\ \\
	\textbf{Ort:} 	\> -- Grosser Einfluss auf Gruppenenergie. Wird oft übersehen.\\ 
					\> -- Platz: 8-10 $m^2$/Person (bis ca. 10) \\ 
					\> -- ggf. Raum selber einrichten \\ \\ 
	\textbf{Ziel:} 	\> -- Zustand am Ende der Moderation aus der Absicht des Auftraggebers ableiten \\ 
					\> -- Erwartete Ergebnisse (auch negative Ziele) \\ \\
	\textbf{Teilnehmer:}	\> -- kann hierarchisch gemischt sein \\ 
							\> -- ggf.: Entscheidungsregeln klären \\ 
							\> -- einzeln durchbesprechen \\ 
							\> -- Anzahl: 8-12 = Optimum \\ 
							\> -- Bei grossen Gruppen: Teilgruppen! 
\end{tabbing}

\section{Was bedeutet Moderation?}

Moderation ist die bewusste Trennung von Prozessebene (wie gehen wir vor?) und Inhalt. Der Prozess, die Arbeitsmittel und -methoden sollten gruppenorientiert und ''gehirngerecht'' sein.

\section{Blitzlicht-Regeln}

Das Blitzlicht ist eine Technik, die mit wenig Vorbereitung und fast ohne Arbeitsmaterial durchgeführt werden kann. Während der Blitzlichtrunde äussern sich die Teilnehmer mit ein bis zwei Sätzen – nicht länger als eine Minute – zu einer vom Moderator gestellten Frage.

\begin{itemize}
	\item Jeder beantwortet die Frage für sich
	\item Keine Bezüge, keine Diskussionen
	\item Blitzlicht, kein Film
	\item Enden mit ''Punkt'' (z.B. der für mich wichtigste Punkt heute war...)
\end{itemize}

\section{Schlüssel-Elemente der Moderation}

\begin{enumerate}
	\item Setting (Halbkreis, keine Tische, Material)
	\item Themenneutrale Moderatoren
	\item Bewusste Trennung zwischen Prozess $\Longleftrightarrow$ Inhalt
	\item Fundierte Vorbereitung (durch Moderator)
	\item Visualisierung aller Ergebnisse
	\item Alle haben freien Zugang zum gemeinsamen Resultat
	\item Teilnehmer sind gleichrangig (können aber unterschiedlichen Einfluss haben)
	\item Klare Ziele, Spielregeln, Entscheidungsregeln
	\item Bewusstes Zeitmanagement
	\item Zweckmässige Umgebung
	\item Häufige Feedbackphasen
	\item Selbstverantwortung jedes Teilnehmers
\end{enumerate}