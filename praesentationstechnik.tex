\chapter{Präsentationstechnik}
Es ist wichtig, dass wir präsentieren können. Schlussendlich wollen wir immer unsere Lösung verkaufen. Wenn wir diese nicht präsentieren können, kauft diese auch niemand. Man muss \textbf{überzeugen}. Dabei spielt es keine wesentliche Rolle ob wir für einen, 12 oder 300 Zuhörer präsentieren.

\section{Vorgehensweise Präsentation vorbereiten (Grundsätze) [MEP]}
Nachfolgende Punkte dienen als Checkliste und können in dieser Reihenfolge abgearbeitet werden.

\begin{itemize}
	\item Aufgabenstellung klären (Briefing) \\
	Was ist das Thema? Ziel der Präsentation? Was ist der Anlass? Wer ist das Publikum? Was ist der Zeitrahmen? Welche Infrastruktur steht zur Verfügung?
	
	\item Material sammeln \\
	Wichtige Aspekte des Themas. Bilder, Zitate, Gegenstände, Handouts, Geschichten, Anekdoten, Versuchsergebnisse, Literatur. Bekannte Meinungen zum Thema. Gibt es Tabu-Zonen?
	
	\item Storyline entwickeln\\
	Man muss sich fragen wo das Publikum vor der Präsentation steht und wo man es nach der Präsentation haben will. Dient später als \textbf{Hilfsmittel}, der rote Faden, um das Publikum zu führen.
	
	\item Abschluss / Finale bestimmen\\
	Der letzte Eindruck bleibt am längsten. Vielleicht Bezug zum Titel der Präsentation, ein Film, eine Anekdote oder "die Moral der Geschichte ist ...", Metaphern. Musser aber zur Storyline passen!
	
	\item Einleitung / Aufhänger am Anfang bestimmen\\
	Interesse, Neugier, Aufmerksamkeit wecken. Teilnehmer positionieren! Frage ans Publikum, Bezug zum Titel, Facts, Geschichte ...\\
	Oft wird folgendes Muster empfohlen. Say, what you're gonna say. Say it. Say, what you have said. Dies wirkt aber oft ermüdend! Dies vorsichtig verwenden und nicht unnötige repetieren.
	
	\item Präsentationsmaterial ausarbeiten\\
	Auf Papier beginnen die Storyline zu entwickeln. Einteilung der Zeit (2-3 min pro inhaltliche Folie). Eventuell Handout machen. Schriftart und Schriftgrösse definieren.
	
\end{itemize}

\section{Merksatz}
\begin{itemize}
	\item Man muss nicht nur keine Idee haben, man muss auch unfähig sein, sie zu präsentieren.
	\item Es geht nicht darum, etwas zu sagen, sondern gehört zu werden!
\end{itemize}

\section{Eine Präsentation gut halten [MEP]}
\begin{itemize}
	\item Eine Präsentation ist ein Dialog (auch wenn es weitgehend schweigt).
	\item Der Teilnehmer will geführt werden.
	\item Halte das Publikum nicht für dümmer als es ist (Respekt und Ernsthaftigkeit ergibt Akzeptanz und Aufmerksamkeit).
	\item Nonverbale Signale der Teilnehmer beachten (Augenkontakt halten). 
	\item Störungen haben Vorrang (bei Störung hat der Vortrag keine Chance).
	\item Kämpfe nie gegen die Teilnehmer.
	\item Kenne Deinen Stoff!
\end{itemize}

\section{Tipps}
\begin{itemize}
	\item Vergleiche machen\\
	Wird beispielsweise mit grossen Zahlen gearbeitet, dann lohnt es sich ein gut vorstellbares Vergleichsmass zu nehmen. Bsp: 1 Bit = 1 Reiskorn, 1 CD = 750 MB = 180 T Reiskörner.
	\item Komplexe Strukturen (komplexe Organigramme)
	Eine komplexe Grafik kann nach und nach eingeblendet werden. Auf der ersten Folie nur die erste Hierarchie Stufe, auf der zweiten die nächste usw.
	\item Präsentation zu Hause trocken üben?\\
	Kann für ungeübte hilfreich sein um ein gutes Timing zu bekommen. Die Gefahr besteht, dass die Präsi dann nicht mehr so frisch sondern einstudiert wirkt.
	\item Folien überspringen?\\
	Oft ist es wichtig den Zeitrahmen einzuhalten und da kann es Sinn machen Folien zu überspringen. Zuvor muss man aber wissen welche und dazu vielleicht nur 1-2 Sätze sagen. Eventuell existiert dazu ein Handout.
	\item Animierte Folien verwenden?\\
	Komplexe Sachverhalte können oft nur gescheit mit Animationen präsentiert werden. Diese aber nicht als Spielerei missbrauchen!
	\item Inhalt der Folie vorlesen?\\
	Hängt vom Stil der Folie ab. Präsentationen sind keine Vorlesungen. Es gibt keine einheitliche Aussage dazu in der Praxis.
	\item Texte oder nur Stichworte?\\
	Zunehmend nur noch Bilder und Stichworte. Erfordert jedoch mehr Ausführlichkeit.
	\item Wie bereite ich mich auf eine wichtige Präsentation vor?\\
	Überraschungen vermeiden und Sicherheiten aufbauen. Redundante Infrastruktur, unsichere Technik vermeiden, Essen / Trinken / WC zuvor. Schöne und gemütliche Kleider. Inhalt kennen! Gelassenheit, Selbstvertrauen!
\end{itemize}

