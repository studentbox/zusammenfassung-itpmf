\chapter{Projekt Controlling}

Vor dem eigentlichen Controlling muss der Projektplan (Ziel $\rightarrow$ Resultat $\rightarrow$ Teilresultat $\rightarrow$ Arbeitspaket) erstellt und die Arbeitspakete (zwischen 5-20 Personentage) geschätzt werden. Im Projektplan werden gleichzeitige APs zu Phasen zusammengefasst, welche sich an den Projektergebnissen orientieren. Für jedes Phasenende wird ein Meilenstein festgelegt. 
Liegt der Projektplan vor kann dieser regelmässig mit dem Ist-Zustand des Projektes verglichen werden. Dieser Abgleich nennt man Projektcontrolling. Das Projektcontrolling hat folgende Ziele:
\begin{itemize}
	\item Projekt in der Zeit und im Budget?
	\item Nächster Meilenstein erreichbar?
	\item Risiken frühzeitig erkennen und bei mehr Aufwand in einem AP Ressourcen verteilen
	\item Verbessern der Schätzungen durch ''Post Mortem'' Analyse.
\end{itemize}
Werden diese Ziele nicht erreicht muss etwas unternommen werden. Beim Projektcontrolling wird ein Soll-/Ist-Vergleich auf der Ebene der Arbeitspakete durchgeführt. Dabei wird jedes AP mit einer Status-Information (offen, in Arbeit, erledigt) versehen. Die Schätzung (SOLL) darf nicht mehr verändert werden. Zieht man das IST vom SOLL ab bleibt das Restbudget übrig (So viel dürfen wir noch verbrauchen...). Bei der Schätzung sollte man nie nach dem Fertigstellungsgrad fragen, sondern nach dem Restaufwand (RAS). SOLL - IST - RAS = positives oder negatives Gesamtergebnis. Kleinere Abweichungen können ausgehalten werden. Bei grösseren Aufwandserhöhungen müssen aber Ressourcen von andern AP umverteilt werden. Bei Software-Entwicklungsprojekten wird üblicherweise alle zwei Wochen ein Projektcontrolling durchgeführt. Bei längeren Projekten auch monatlich oder in kritischen Phasen auch wöchentlich.
