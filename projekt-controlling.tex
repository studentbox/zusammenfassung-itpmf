\chapter{Projekt Controlling}

\section{Voraussetzung}
Vor dem eigentlichen Controlling muss der Projektplan (Ziel $\rightarrow$ Resultat $\rightarrow$ Teilresultat $\rightarrow$ Arbeitspaket) erstellt und die Arbeitspakete (zwischen 5-20 Personentage) geschätzt werden. Im Projektplan werden gleichzeitige APs zu Phasen zusammengefasst, welche sich an den Projektergebnissen orientieren. Für jedes Phasenende wird ein Meilenstein festgelegt. 
Liegt der Projektplan vor kann dieser regelmässig mit dem Ist-Zustand des Projektes verglichen werden. Dieser Abgleich nennt man Projektcontrolling. 

\section{Ziele}
\begin{itemize}
	\item Projekt in der \textbf{Zeit} und im \textbf{Budget}?
	\item Nächster \textbf{Meilenstein} erreichbar?
	\item \textbf{Risiken} frühzeitig erkennen, falls geschätzte Gesamtaufwände massiv steigen und fortgesetzt werden.
	\item \textbf{Aufwandsverschiebung} zwischen APs: Ressourcen umverteilen.
	\item Verbessern der Schätzungen durch \textbf{''Post Mortem''} Analyse. (Post Mortem = Nach dem Tod. Schätzungen reflektieren und prüfen ob man gut geschätzt hat. Sonst Gründe suchen und Schätzmethode verbessern).
\end{itemize}
Werden diese Ziele nicht erreicht müssen Massnahmen ergriffen werden um das Projekt in die richtige Richtung zu lenken (steuern).

\section{Verfahren}
Beim Projektcontrolling wird ein Soll-/Ist-Vergleich durchgeführt. Je nach grösse des Projektes arbeitet man auf Stufe der Teilresultate oder Arbeitspakete. Dabei wird jedes AP mit einer Status-Information (offen, in Arbeit, erledigt) versehen. Die Schätzung (SOLL) darf nicht mehr verändert werden. Zieht man das IST vom SOLL ab bleibt das Restbudget übrig (So viel dürfen wir noch verbrauchen...). Bei der Schätzung sollte man nie nach dem Fertigstellungsgrad fragen, sondern nach dem Restaufwand (RAS). SOLL - IST - RAS = positives oder negatives Gesamtergebnis. Kleinere Abweichungen können ausgehalten werden. Bei grösseren Aufwandserhöhungen müssen Ressourcen von andern AP umverteilt werden. Bei Software-Entwicklungsprojekten wird üblicherweise alle zwei Wochen ein Projektcontrolling durchgeführt. Bei längeren Projekten auch monatlich oder in kritischen Phasen auch wöchentlich.


\section{Sie können erläutern, warum Projektcontrolling betrieben wird?}
Oberstes Ziel: Erfolgreicher Projektabschluss. Regelmässiger Abgleich des aktuellen Stands mit dem Plan (Ist-Soll). Dies ist einer der wichtigsten Projekt-Steuerungs Instrumente. Mit diesem Informationen müssen anschliessend Massnahmen ergriffen werden.

\section{Sie sind in der Lage, auf Grundlage einer Liste von Arbeitspakten ein sauberes Controlling für ein Software-Projekt aufzusetzen.}
Man hat alle Arbeitspaket mit der initialen Schätzung (in der Regel in PT) und deren Status (open, in work, done). Dann nimmt man den Ist-Stand auf (heute) - wie viel PTs wurden pro Arbeitspaket verbraten. Dann macht man eine Umfrage wie viel Restaufwand noch pro AP benötigt werden. Nun berechnet man SOLL-IST-RAS und hat das Ergebnis. Und nun Aufwände umverteilen oder andere Massnahmen ergreifen. Man kann immer am Scope, Zeit oder Budget schrauben (in der Regel eine Kombination nach Genehmigung). Achtung, wenn ich zu einem Team von 6 Entwickler 2 hinzufüge, steigt die Produktivität nicht um 1/3 sondern weniger.