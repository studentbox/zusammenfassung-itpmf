\chapter{Project Setup \& Planning}

\section{Sie sind in der Lage, begründet darzulegen, worauf es bei dem Projekt-Setup ankommt.}

Ein Projekt ist ein zeitlich begrenztes Vorhaben mit einem definierten Anfang und Ende. Weil in den einzelnen Projektphasen viele Fragen auftauchen (Was ist das Problem?, Wie sehen die Prozesse aus? usw.), braucht es vor dem Projekt ein Projektsetup welche den Zeitrahmen, die Ziele und Resultate des Projektes festlegt.

\section{Sie können erläutern, was ein Projektziel ist und sind in der Lage, anhand einer Situationsbeschreibung korrekte Projektziele zu definieren.}

Projekt-Ziele beschreiben was man mit dem Projekt will. Folgende Fragen führen zu Projekt-Zielen:
\begin{itemize}
	\item Was soll erreicht werden?
	\item Welche Eigenschaften soll der Endzustand haben?
	\item Was ist nach dem Projekt anders - besser! - als vorher?
\end{itemize}
Projekt-Ziele sollten messbar und beurteilbar sein jedoch keine Lösung aufzeigen. Tipp: Zuerst Resultate beschreiben und danach die Ziele!

\section{Sie vermögen den Unterschied zwischen Projekt-Zielen und Projekt-Ergebnissen zu erklären.}

Von den Projekt-Zielen werden die Projekt-Ergebnissen/Resultate abgeleitet und von den Resultaten die einzelnen Arbeitspakete. Die Resultate sind die Ergebnisse die ein Projekt erzeugt. Resultate sind kein Aktivitäten (Nicht das Durchführen eines Workshops ist interessant, sondern z.B. die daraus hervorgegangenen Anforderungen.)

Ergebnisse in Softwareprojekten lassen sich fast immer in folgende Kategorien unterteilen:

\begin{enumerate}
	\item Applikationssoftware
	\item Applikationsdokumentation
	\item Prozesse
	\item Migration
	\item Rollout
	\item Projektmanagement-Ergebnisse
\end{enumerate}