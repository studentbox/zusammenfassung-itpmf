\chapter{Kommunikation im Projekt}

\section{Sie kennen die Bedeutung und wichtigsten Elemente der Kommunikation im Projekt}

\subsection{Projektführungselemente}
\begin{description}
	\item[Interne]  Projektplanung (Resultate, AP, Aufwände, Meilensteine), Projektsitzung (Koordination, Abstimmung), TP-Review, Einzelgespräche / Stichproben.
	\item[Externe] Projektreport, Stakeholder-Management, Risiko-Management.
\end{description}

\subsection{Wichtigsten Fähigkeiten eines Projektleiters}
\begin{itemize}
	\item Leadership
	\item Kommunikation
	\item Problem-Lösung
	\item Verhandlung
	\item Beinflussung der Organisation (Organisational Change Mgmt.)
	\item Mentoring (Tutoren prädestiniert!)
	\item Prozess- and technische Expertise
\end{itemize}

\subsection{Wichtigsten Kommunikations-Werkzeuge}
\begin{description}
	\item[Projektsitzung] Kommunikation INNERHALB des Projekts.
	\item[Projektreport/-statusbericht] Regelmässige Kommunikation nach AUSSEN.
	\item[Stakeholder-Management] Gezielte Einflussnahme (Lobbying) auf Nutzniesser und Beteiligte bzw. Betroffene des Projekts.
\end{description}


\section{Sie sind in der Lage das Stakeholder Management zu erläutern}
Der Projektleiter kommuniziert mit Sponsoren, Projektteam, Kunden, Nutzern, Contractors und Lininenmanager. Stakeholder sind Anspruchsgruppe und -personen, die unmittelbaren Einfluss auf den Projektfortschritt haben und/oder von den Projektzielen direkt oder indirekt betroffen sind. Bsp: Finanzchef, Auftraggeber, Kunden, Gesetzgeber. Definition Stakeholder nach Chris Rupp: 
\begin{quote}
	Ein Stakeholder eines Systems ist eine Person oder Organisation, die direkt oder indirekt Einfluss auf die Anforderung des betrachteten Systems hat.
\end{quote}

Es gibt 4 Stakeholder-Gruppen:
\begin{description}
	\item [Promotoren, Sponsoren] aktive Unterstützuugn, Erfolg sicherstellen, Commitment, positive Energie, Macht, Geld, Ressourcen bereitstellen, bei Verlust schwerwiegende Folgen für Projektfortschritt
	\item [Supporters, Change Agents] inhaltliche Unterstützung, breite Abstützung in Organisation, punktuell Ressourcen bereitstellen.
	\item [Opponents, Change Barriers] Offene oder heimlicher Widerstand gegen das Projekt, wesentlicher negativer Einfluss auf Projektziele. Ziel: Projektabbruch, Umbesetzung von Schlüsselfunktionen, Aneignung des Projekts.
	\item [Hoppers, Change Advocates] unentschlossen, neutral, Stellung wechselnd, kein direkter Machteinfluss, aber Meinungsbildung möglich, durch Massnahmen => Supporters.
\end{description}

Bei einer Stakeholderanalyse muss man wie folgt vorgehen:
\begin{enumerate}
	\item Identifizieren der Stakeholder (Ergebnis: Stakeholderliste/map)
	\item Beziehungsanalyse (Ergebnis: Stakeholdermap mit Beziehungen)
	\item Kommunikation analysieren (Ergebnis: Kommunikationsplan)
\end{enumerate}
	
Pro Stakeholder folgendes festhalten: Rolle, Name, Funktion, Auftrag, Ziele, Chance/Interessen, Risiken/Konfliktpotentiale und Massnahmen.

\section{Sie sind in der Lage eine Projektstandsitzung zu gestalten}
Wichtigstes Führungsinstrument innerhalb des Projekts. Die Agenda für eine Sitzung kann etwa so aussehen:

\begin{enumerate}
	\item Letztes Protokoll (formale Abnahme des letzten Protokolls)
	\item Allg. Infos (Organisatorische Veränderungen, Budgetpläne)
	\item Status u. Querinfos (Aktueller Stand der Resultate mit Termin, Restaufwand, Status)
	\item Pendenzen (kleinere Aufträge an Teilnehmer)
	\item Status der Issues (wichtige Fragestellungen aus dem Projekt, welche für das Projekt zentral sind)
	\item Risiken (Eintrittswahrscheinlichkeit, Auswirkungen, Massnahmen)
	\item Nächste Termin (für die nächsten 3 Monate)
	\item Spezial und Fachthemen (kein Workshop!)
	\item Ferienliste/Abwesenheiten
	\item Diverses (max. 5 Minuten)
\end{enumerate}

Die Projektstandsitzung bietet einen regelmässigen Informationsfluss innerhalb des Projekts. Übt einen gewissen Druck auf die Projekt-Mitarbeiter aus im Bezug auf die Resultate. Plattform um über Probleme zu sprechen, zu lösen und Entscheide zu fällen. Reporting der Projekt-MA und Teilprojektleiter gegenüber dem Projektleiter.

Agenda muss klar und fix sein. Wöchentlich oder 2x wöchentlich. Teilnehmer: Projektleiter, Teilprojektleiter oder Resultat-Verantwortliche. Dauer: max. 90-120 min.

\section{Sie können einen Projektreport gestalten und beurteilen}
Der Projektreport oder auch Projekt-Statusbericht dient als regelmässige Kommunikationsinstrument nach AUSSEN - gegenüber Auftraggeber, STC-Mitglieder, wichtigen Stakeholder, Kunde und Nutzer. Periode: Monatlich oder in intensiven Phasen wöchentlich möglich.

\subsection{Quellen}
Massnahmen-Katalog, Open Task Liste, Terminplan, Ressourcenplan, Lieferobjekte, Probleme, Risikokatalog, Projektkostenplan. Dies sind die Quellen für den Report, der PL ist darum besorgt die Quellen aktuell zu halten.

\subsection{Struktur u. Inhalt}
\begin{description}
	\item [Header] Berichtsperiode, Datum, PL, Autor
	\item [Projekt Status] Mit Ampeln arbeiten für die folgende Themen: Overall, Kosten, Zeit, Ressourcen, Focus und Interfaces. Zu jeder Ampel eine Info zusätzlich zu roten Ampel die Massnahmen. Man darf auch Erfolgsmeldungen melden.
	\item [Status Kosten] Soll-Ist
	\item [Status der Projektresultate] Next-Steps und Resultate Statis (Soll-Ist)
	\item [Risiken (optional)]
\end{description}
