\chapter{Strategische IT-Führung}

\section{Governance II}
\subsection{Governance Entscheidungsarchetypen}
\subsubsection{Business Monarchie}
\textbf{Vorteile}
\begin{enumerate}
	\item Entscheidungen sind am Businessbedarf orientert
	\item Verantwortung des Business ist höher
	\item Schnelle Entscheidung
	\item Gute Umsetzungsunterstüzung vom Business
\end{enumerate}
\textbf{Nachteile}
\begin{enumerate}
	\item Zu wenig IT Knowhow
	\item Zu kurzfristig orientiert
	\item Zu kostenorientiert (zu wenig Investitionsbereit)
\end{enumerate}
\subsubsection{IT Monarchie}
\textbf{Vorteile}
\begin{enumerate}
	\item Technologisch gute Entscheidungen
	\item Auf IT Infrastruktur passende Lösungen
	\item Schnelle Trendserkennung
	\item Schnelle Entscheidung \& Umsetzung
\end{enumerate}
\textbf{Nachteile}
\begin{enumerate}
	\item Zu wenig kostenorientiert
	\item Technologieverliebte Entscheidungen, Einkauf unnötiger cooler Systeme
	\item Zu wenig Businessfokus, zu viel auf IT
\end{enumerate}
\subsubsection{Feudalistisch}
\textbf{Vorteile}
\begin{enumerate}
	\item Perfekt auf den Bereich zugeschnittene Lösungen
	\item Entscheidungen sind lokal verankert
	\item Schnelle Entscheidungen, da keine Abstimmung notwendig, keine Streitereien, jeder darf das haben was er haben will
	\item Hohe Dynamik
\end{enumerate}
\textbf{Nachteile}
\begin{enumerate}
	\item Heterogene Architektur
	\item Kostenintensive Integration
	\item Redundante Systeme
	\item Schlechter Informationsfluss
	\item Inseldenken
	\item Kaum übergreifende Lösungen -> IT Silos
	\item Kein Wissenstransfer
	\item KnowHow intensive IT
	\item Unternehmensweit unterschiedliches IT Wissen dass man nutzen muss
\end{enumerate}
\subsubsection{Föderalistisch}
\textbf{Vorteile}
\begin{enumerate}
	\item Breite Abstützung der Entscheidung
	\item Einheitliche Umsetzung ist einfacher
	\item Know How Transfer findet statt
	\item Hoher Meinungsaustausch, verschiedene Sichten, differenzierte Entscheidung möglich
\end{enumerate}
\textbf{Nachteile}
\begin{enumerate}
	\item Lange Entscheidungsprozesse
	\item Gefahr der Blockade einer Entscheidung - vorsichhenschieben der Entscheidung durch Grabenkämpfe, Krisen, gegenseitiges Blockieren...
	\item Teure Lösungen bei fehlenden Kompromissen
\end{enumerate}
\subsubsection{IT Duopol}
\textbf{Vorteile}
\begin{enumerate}
	\item Technisch und geschäftlich optimale Lösungen
	\item Transfer von Wissen zwischen Business \& IT
	\item Verbindung von Technologietrends \& Businessinnovation
	\item Gute Umsetzung
\end{enumerate}
\textbf{Nachteile}
\begin{enumerate}
	\item Verständnis \& Kommunikationsprobleme
	\item Gegenseitige Behinderung von Entscheidungen
\end{enumerate}
Unbedingt gemeinsame Ziele definieren!
\subsubsection{Anarchie}
\textbf{Vorteile}
\begin{enumerate}
	\item Sehr kurze Entscheidungsfristen (keine Absprachen nötig)
	\item Innovationsfreudig / Kreativ
\end{enumerate}
\textbf{Nachteile}
\begin{enumerate}
	\item Insellösungen / Adhoc Lösungen
	\item Heterogene IT Landschaften
	\item Hohe Wartungskosten
	\item Keine globale Strategie - grosse Investitionen sind nicht möglich
\end{enumerate}

\subsection{IT Governance Fallstricke}
\textbf{Symptome für Governance Probleme}
\begin{itemize}
	\item Bastellöusngen / SchattenIT
	\item Keine IT-Kennzahlen für Leistungsmessung / Bewertung de rIT
	\item Klagen über IT aus Management
	\item Redundante Systeme / Redundante Datenhalten
	\item IT entspricht nicht den Erwartungen
		\subitem Niedrige Kundenzufriedenheit
		\subitem Instabile Systeme
		\subitem Gescheiterte Projekte
	\item Explodierende IT Kosten bei niedrigem ROI
	\item IT Dienstleister nicht im Griff
\end{itemize}

\textbf{Massnahmen}
\begin{itemize}
	\item \textbf{Executive Engagement} \\
	IT Governance muss Chefsache werden
	\item \textbf{Policies as Strategic Tools} \\
		Es sollen klare und sinnvolle IT Prinzipien definiert werden, welche tatäschlich einen Nutzen bringen.
	\item \textbf{Defined Hierarchy of Governing Bodies} \\
		Wer was wie festlegen? Am besten eine Governence Arrangment Matrix aufbauen
	\item \textbf{Delegation of Authority and Precedent} \\
		Nicht zu viel zentralisieren bei den Entscheidungen, auch dezentrale Muster berücksichtigen.
	\item \textbf{Business Alignment} \\
		IT an Businessbedarf ausrichten
	\item \textbf{Proactive Liasion and Communications} \\
		Jeder Geschäftsbereich soll auf Management einen IT Ansprechspartner haben, der auch proaktiv auf sie zugeht und sie informiert.
	\item \textbf{Metrics and Reporting} \\
		Sinnvolle Kennzahlen definieren und erheben
	\item \textbf{Appropriate-Weight Procedures, Standards and Controls} \\
		Entscheidungen werden auf sinnvoller Ebene getroffen.
	\item \textbf{Independent Scrutiny} \\
		Das Reporting / Controlling wird von einer unabhängiger Instanz durchgeführt. Es geht nicht darum, die Leute zu überwachen, sondern um zu sehen ob es jetzt funktioniert.
	\item \textbf{Training and Awareness Program} \\
		Weiterbildung auf Governance Mechanismen
	\item \textbf{Easy-to-use Tools} \\
		Einfache Tools.
\end{itemize}